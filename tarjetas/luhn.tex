El \textbf{algoritmo de Luhm}, también conocido como \textbf{módulo 10} o \textbf{mod 10}, se utiliza para verificar el número de la tarjeta. Fue creado por el ingeniero de IBM \textbf{Hans Peter Luhn} en 1954. Consiste en una suma para validar un número utilizando el operador módulo de 10.

El dígito de control del PAN se cálcula con este algoritmo, los pasos son los siguientes:

\begin{enumerate}
  \item Eliminar el dígito de control, si está presente.
  \item Seleccionar números de derecha a izquiera, del último al primero. Doblar cada segundo dígito. Si el doble es un número mayor que 9, se suman los dígitos o se resta 9 al doble.
  \item Sumar todos los dígitos resultantes, doblados como no doblados.
  \item Verificar el dígito de control con la fórmula $(10-(s \mod 10))\mod 10$, donde $s$ es la suma del paso 3. Este es el número más pequeño, puede ser 0, que debe ser agregado a $s$ para ser múltiplo de 10.
\end{enumerate}

El algoritmo en pseudocódigo se muestra abajo:

\begin{lstlisting}[style=general,caption=Luhn Check Digit]
// suma los numeros de la tarjeta
function luhn_checksum(num) {
  int len = num.length;
  int parity = len % 2; // 0 par, 1 impar
  int sum = 0;
  for (int i = len - 1; i >= 0; i--) {
    int d = parseInt(num.charAt(i));
    if (i % 2 == parity) { d *= 2 }
    if (d > 9) { d -= 9 }
    sum += d;
  }
  return sum % 10;
}

// calcula el digito de control
function luhn_calculate (num) {
  int control = luhn_checksum(num + "0");
  return control == 0 ? 0 : (10 - control) \% 10;
}
\end{lstlisting}
