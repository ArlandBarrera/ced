El precio del dinero en el tiempo.

Se parte del hecho de que, considerando \textbf{ceteris paribus}, las personas prefieren consumir en el presente que consumir en el futuro si la cantidad y calidad a consumir es la misma.

Esto cambia considerando cantidades distintas en el futuro. En este caso se toma en cuenta la \textbf{preferencia temporal}. Se puede preferir consumir en el presente o futuro. Si se consume en el presente la preferencia temporal es alta, por el contrario si se posterga el consumo a futuro la preferencia temporal es baja.

Las sociedades puden ser consumidoras o ahorradoras. Si las personas prefieren consumir el ahorro será bajo, por el contrario, si prefieren ahorrar el ahorro será alto. La oferta de dinero en el tiempo determina la tasa de interés. Si se prefiere el consumo, si la preferencia temporal es alta, la tasa de interés es alta. Si se prefiere el ahorro, si la preferencia temporal es baja, la tasa de interés es baja.

Si la oferta de ahorro es alta la tasa de interés baja, porque hay suficiente oferta de ahorro para cubrir la demanda de ahorro, por tanto el precio del ahorro baja.

Si la oferta de ahorro es baja la tasa de interés sube, porque no hay suficiente oferta de ahorro para para satisfacer la demanda de ahorro, por tanto el precio del ahorro sube.

Cuando la tasa de interés es baja se fomenta el financiemento o las inversiones a largo plazo, lo cual hace que la economía florezca.

Cuando la tasa de interés es alta hay menos ahorro que se pueda destinar a financiemento o inversiones, además es posible que la tasa de interés supere las beneficios esperados, al menos durante un tiempo, de la inversión lo cual hace la inversión menos atractiva. Esto fomenta las inversiones a corto plazo, compra y venta rápidda en breves periodos de tiempo.

La tasa de interés se refiere al porcentaje (\%), un valor porcentual, y el interés, como tal, se refiere a un valor numérico.

En cálculos o ecuaciones se utiliza el valor decimal, el valor del interés. Este valor se obtiene dividiendo el valor porcentual entre 100. Se efectua la siguiente operación:

\[\boxed{
  \text{valor decimal} = \dfrac{\text{valor porcentual}}{100}
}\]

La tasa de interés $r$ se expresa en porcentaje. Para ello se multiplica el valor decimal por 100. Se realiza la siguiente operación:

\[\boxed{
  \text{valor porcentual} = \text{valor decimal}*100
}\]

Por ejemplo, se tiene 7.5\% y se quiere su valor decimal:

\[
  7.5\% = \dfrac{7.5}{100} = 0.075
\]

Otro ejemplo, se desea expresar en porcentaje el valor 0.639:

\[
  0.639*100 = 63.9\%
\]
