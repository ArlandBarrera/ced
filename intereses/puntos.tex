Un \textbf{punto porcentual} se utiliza para expresar la \textbf{diferencia absoluta} entre dos valores porcentuales. Equivale directamente a la centésima parte de un valor, es decir, al 1\% de dicho valor.

Se emplean fundamentalmente para cuantificar la variación o el cambio entre dos porcentajes, especialmente en análisis estadísticos y económicos.

\textbf{Ejemplo:} Si la tasa de aprobación en una encuesta sube del $32\%$ al $48\%$, el incremento es de 16 puntos porcentuales ($48\% - 32\% = 16$).

Un \textbf{punto básico} (a menudo abreviado como \textbf{pb} o \textbf{BPS}) es una unidad de medida más fina, equivalente a la centésima parte de un punto porcentual.

Esto implica la siguiente relación: 1 punto básico = 0.01\%.

Por lo tanto, 100 puntos básicos = 1 punto porcentual (1\%).

Su uso es crucial en el ámbito financiero para expresar con precisión cambios muy pequeños en tasas de interés, rendimientos de bonos o tipos de cambio, evitando así cualquier ambigüedad.

\textbf{Ejemplo:} Si una tasa de interés asciende del 0.25\% al 0.50\%, el aumento es de 0.25 puntos porcentuales. Para mayor precisión, se comunica que el incremento es de 25  puntos básicos ($0.50\% - 0.25\% = 0.25\%$).
