En la sección sobre el interés compuesto se detalló como la frecuencia de periodos $n$ afecta el valor total. Mientras más frecunte sea la acumulación, mayor es el valor del monto acumulado exponencialmente. En este sentido, si se pudiese acumular en cada momento posible se consigue una acumulación continua obteniendo un interés continuo. Esto maximisa la ganancias.

En 1683, el matemático suizo Jacob Bernoulli estaba estudiando el interés compuesto y uno de los ejemplos que consideró es el siguiente: un monto inicial de \$1 con una tasa de interés del 100\% por año. Considerando la ecuación \ref{equintcmp}, se obtiene la siguiente expresión:

\[
  \left(1+\frac{1}{n}\right)^n
\]

Jacob se interesó por saber cual sería el máximo beneficio posible generado por los intereses. Esto se consigue cuando $n$ alcanza un valor muy grande. La expresión para esto sería el siguiente límite:

\[
  \lim_{n\to\infty} \left(1+\dfrac{1}{n}\right)^n
\]

Jacob Bernoulli descubrió que el número al cual tiende este límite es 2.718281 . . . , y aunque esta constante matemática fue introducida por él, el primero en usar la letra e para referirse a ella fue el matemático suizo Leonhard Euler en su obra Mechanica (1736).

\[\boxed{
  \lim_{n\to\infty} \left(1+\dfrac{1}{n}\right)^n = e
}\]

De lo anterior se puede concluir que las ganancias que resultan del interés compuesto no son infinitas, si no que al tender a valores grandes se realentizan y se obtiene el número e. Por tanto, se obtiene una expresión que maximisa el interés compuesto:

\begin{listequbox}
  {A=Pe^{rt}}{equintcont}{Interés continuo}
\end{listequbox}

\textbf{En la práctica el interés continuo no es utilizado, dado que acumular intereses en cada momento posible no es una operación viable.}
