Se considera un precio total $T$, del cual se tienen $C$ cantidad de unidades. El  precio unitario $U$, el precio de cada unidad individual, es la razón entre el precio total y la cantidad de unidades. Esto se expresa de la siguiente forma:

\[\boxed{
  U = \dfrac{T}{C}
}\]

Elementos:

\begin{itemize}
  \item \textbf{T}: Precio total, el costo por toda la cantidad adquirida.
  \item \textbf{C}: Cantidad total de unidades a comprar.
  \item \textbf{U}: Precio unitario, el costo una única unidad.
\end{itemize}

\textbf{Ejemplos:}

\textbf{1)} Un paquete de 16 GB de paquetes de datos cuesta \$35. Cuál es el precio de un 1 GB?

\begin{align*}
  U &= \dfrac{\$35}{16 \text{GB}} \\
  U &= 2.1875 (\text{\$/GB})
\end{align*}

Cada paquete de datos de 1 GB cuesta \$2.19.

\textbf{2)} Una bolsa de 5 kg de carne cuesta \$40. Cuánto cuesta un 1 kg?

\begin{align*}
  U &= \dfrac{\$40}{5 \text{kg}} \\
  U &= 8 (\text{\$/kg})
\end{align*}

Cada bolsa de carne de 1 kg cuesta \$8.
