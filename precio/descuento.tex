Considerando un monto $P$, un descuento $d$ es la diferencia entre $P$ y un valor descontado. El valor descontado es un porcentaje $r$ de $P$. Esto se expresa de la siguiente forma:

\[
  d = P - Pr
\]

Factorizar:

\[\boxed{
  d = P\left(1 - r\right)
}\]

También se puede expresar como porcentaje:

\[
  d = P\left(100\% - r\%\right)
\]

\textbf{Ejemplos:}

\textbf{1)} 30\% de descuento de \$4.99

\begin{align*}
  d &= 4.99\left(1 - 0.3\right) \\
  d &= 4.99\left(0.7\right) \\
  d &= 3.493
\end{align*}

Redondear el precio: $\$3.50$

\textbf{2)} 70\% de descuento de \$12.75

\begin{align*}
  d &= 12.75\left(100\% - 70\%\right) \\
  d &= 12.75\left(30\%\right) \\
  d &= 3.825
\end{align*}

Redondear el precio: $\$3.82$
