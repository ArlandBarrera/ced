Considerando un monto $P$, este valor puede aumentar o disminuir en base a un aumento o descuento. Genenarlemte estos valores descontados o aumentados se expresan en porcentajes. Un nuevo valor $P_F$, es resultado de tomar un porcentaje $r$ de $P$ y sumar o restar depenediendo si es aumento o descuento. Esto se expresa de la siguiente forma:

\[
  P_F = P \pm Pr
\]

Factorizar $P$ en la expresión:

\[\boxed{
  P_F = P\left(1 \pm r\right)
}\]

Sumar es para aumento y restar para descuento.

\textbf{Aumento}

\begin{listequbox}
  {P_F = P(1 + r)}{equaumento}{Aumento}
\end{listequbox}

\textbf{Descuento}

\begin{listequbox}
  {P_F = P(1 - r)}{equdescuento}{Descuento}
\end{listequbox}

\textbf{Ejemplos:}

\textbf{1)} 30\% de descuento de \$4.99

\begin{align*}
  P_F &= 4.99\left(1 - 0.3\right) \\
  P_F &= 4.99\left(0.7\right) \\
  P_F &= 3.493
\end{align*}

Redondear el precio: $\$3.50$

\textbf{2)} 70\% de descuento de \$12.75

\begin{align*}
  P_F &= 12.75\left(100\% - 70\%\right) \\
  P_F &= 12.75\left(30\%\right) \\
  P_F &= 3.825
\end{align*}

Redondear el precio: $\$3.82$
