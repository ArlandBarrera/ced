La \textbf{Secretaría de Hacienda y Crédito Público} (SHCP) es la encargada de diseñar, planear, ejecutar y coordinar las políticas públicas en materia de economía y finanzas públicas. Surgió en 1821, uno de los primeros ministerios al crearse la nación independiente.

El \textbf{Servicio de Administración Tributaria} (SAT), creado en 1997, tiene la responsabilidad de aplicar la legislación fiscal y aduanera con el fin de que las personas físicas y morales contribuyan proporcionalmente y equitativamente al gasto público y de fiscalizar a los contribuyentes para que cumplan con las disposiciones tributarias y aduaneras.

El \textbf{Impuesto Sobre la Renta} (ISR) aplca tanto a personas físicas como a personas morales (empresas). Se calcula aplicando una tasa progresiva a los ingresos obtenidos. En otras palabras, las personas con ingresos más altos pagan una tasa impositiva más alta que aquellas con ingresos más bajos.

El \textbf{Impuesto de Valor Agregado} (IVA) se cobra en cada etapa de la cadena de producción y distribución, desde el fabricante hasta el consumidor final. La tasa general es del 16\% y en la frontera del país es del 8\%.

El \textbf{Impuesto Especial sobre Producción y Servicios} (IEPS) se aplica a ciertos productos específicos como tabaco, bebidas alcohólicas, combustibles, entre otros.

El \textbf{Impuesto Predial} se aplica a la propiedad raíz, ya sea terrenos o construcciones. La tasa varía según la ubicación, tamaño y uso de la propiedad.

El \textbf{Impuesto sobre Nómina} grava los salarios y remuneraciones pagados por las empresas. La base puede incluir salarios, bonificaciones, comisiones y otros beneficios laborales.
