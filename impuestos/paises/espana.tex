El \textbf{Ministerio de Hacienda} (MH) es el departamento de la \textbf{Administración General del Estado} encargado de la propuesta y ejecución de la política del \textbf{Gobierno de la Nación} en materia de haciendia pública, presupuestos, gastos y empresas públicas. Es uno de los primeros ministerios en crearse, siendo su origen en el siglo XVIII con la administración Borbónica.

La \textbf{Agencia Estatal de Administración Tributaria} o \textbf{Agencia Tributaria}, creada en 1992, le corresponde la aplicación efectiva del sistema tributario estatal y aduanero de forma que se cumpla el principio constitucional en virtud del cual todos han de contribuir al sostenimiento de los gastos públicos de acuerdo con su capacidad económica.

El \textbf{Impuesto sobre la Renta de las Personas Físicas} (IRPF) es uno de las más relevantes. Afecta a todas las personad físicas. Se aplica sobre los ingresos obtenidos, ya sea por trabajo (nómina), actividades económicas, inversiones, alquileres u otros. La tributación es progresiva, es decir, a mayor ingreso, mayor tipo impositivo.

El \textbf{Impuesto sobre Sociedades} afecta a las empresas y grava los beneficios obtenidos por la actividad empresarial. El tipo general es del 25\%.

El \textbf{Impuesto sobre el Patrimonio} grava el valor del patrimonio neto de las personas físicas el 31 de diciemnbre de cada año.

El \textbf{Impuesto de Valor Añadido} (IVA). Los tipos vigentes son el general (21\%) para la mayoría de bienes y servicios, el reducido 10\% para alimentos, transporte, hostelería, etc. y el superreducido 4\% para productos de primera necesidad.

Los \textbf{Impuestos a los hidrocarburos} (gasolina, diésel, gas natural) se componen principalmente del \textbf{Impuesto Especial sobre hidrocarburos} (IEH), con tramos fijos, y el IVA (21\%). El tramo estatal, actualmente en torno a 0.4007 \euro /litro para gasolina, y el tramo especial, en torno a 0.072 \euro /litro para gasolina, componen el IEH, cuyo valor fijo es en torno a 0.473 \euro /litro para gasolina. El IVA se aplica sobre el coste del combustible más el IEH, por tanto el precio final es IVA * (combustible + IEH).
