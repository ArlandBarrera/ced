El \textbf{Ministerio de Economía y Finanzas} (MEF) tiene a su cargo todo lo relacionado a la formulación de iniciativas en materia de política económica.

La \textbf{Dirección General de Ingresos} (DGI) se encarga de administrar el sistema tributario panameño. Fue creada mediante la Ley \#8 del 30 de noviembre de 1964. Se estructuró mediante el Decreto Gabinete 109, del 7 de mayo de 1970.

El \textbf{Registro Único de Contribuyente} (RUC) es el número de identificación tributaria, en caso de ser persona jurídica se debe solicitar. Para una persona natural, este es el número de cédula de identidad personal.

El \textbf{Número de Identificación Tributario} (NIT) es la clave de accedo para gestionar en línea los trámites y servicios ofrecidos por la DGI.

El \textbf{Dígito Verificador} (DV) es un número de identificación que el sistema asigna de manera automática a cada RUC, sea persona natual o jurídica. Se usa al momento de hacer algún pago a la DGI.

El \textbf{Impuesto de Transferencia de Bienes Muebles y Servicios} (ITBMS) equivale al IVA en otros paises. Equivale a una tarifa general del 7\%, se aplica a la mayoria de los productos.
