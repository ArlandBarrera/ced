En ocaciones es conveniente trabajar con una parte de un valor, no con todo el valor. Una forma de expresar esto es el porcentaje. Esto es, una \textbf{porción} de un valor dividido en \textbf{cien}. Es una razón expresada como una fracción de 100. La palabra porcentaje deriva de la frase en latín \textit{per centum}, que significa "por ciento". Es denotado con el símbolo \%.

En la antigua roma en ocaciones los cálculos se hacían en fracciones en los múltiplos de $\frac{1}{100}$. Por ejemplo, el emperador Augusto estableció un impuesto conocido como \textit{centesima rerum venalium} que dictaba que había que pagar el $\frac{1}{100}$, esto es el 1\%, sobre los bienes vendidos en subastas. Para facilitar los cálculos utilizaban fracciones simplificadas a las centenas.
