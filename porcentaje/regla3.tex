Se requieren 3 valores conocidos para aplicar una regla de 3, con el objetivo de encontrar un cuarto valor.

Valores:

\begin{itemize}
  \item \textbf{a:} valor total
  \item \textbf{b:} valor parcial
  \item \textbf{x:} porcentaje total (\%)
  \item \textbf{y:} porcentaje parcial (\%)
\end{itemize}

Los valores valores porcentuales para poder ser utilizados en cálculos tienen que ser convertidos a su forma decial, esto se consigue dividiendo un valor porcentual entre 100.

\[\boxed{
  \dfrac{a}{b}=\dfrac{x}{y}
}\]

El valor que se busca se despeja a conveniencia, por ejemplo si se busca b:

\[
  b=\dfrac{ay}{x}
\]

\textbf{Ejemplo:}

Se desea conocer el precio de una prenda que originalmente costaba \$125 y ahora su precio es el 35\% del valor orignal.

Reemplazar:

\[
  b=\dfrac{\$125*35\%}{100\%}
\]

Convertir valores porcentuales en decimales:

\[
  b=\$125*0.35
\]

Resultado:

\[
  b=\$43.75
\]

El 35\% de \$125 es \$43.75.
